\chapter{Conclusion}
\label{cha:conclusion}


Knowledge graph completion is one of essential tasks since most available knowledge graphs are often missing many facts, and some of the edges they contain are incorrect \cite{angeli2013philosophers}. There are two approaches to adding missing relation \cite{wang2017knowledge}: entity prediction and relation prediction. However, the former have more attention than the latter \cite{chang2020benchmark}.
Therefore, the main objectives of this thesis is to study the impact of relation prediction in training objective and in evaluation protocol.  

In this study, in order to incorporate the relation prediction into training objective, instead of only replacing the subject or objects in a triple to generate negative example (standard training objective), we also include the negative triples that generated by replacing the true relation with the relations in KG. We denoted that training as hybrid training objective. 

Furthermore, to investigate the impact of relation on evaluation protocol, instead of selecting best-performance models on entity MRR, we investigate the effect of using overall MRR where the micro-average between MRRs of subject prediction, object prediction and relation predictions. 

Within the limited of time, we can only study the impact of hybrid training on ComplEx mode, however, the finding led to a nice outcome, hybrid training does bring positive effect not only to entity prediction, but also to relation prediction. 

Furthermore, by selecting models on overall MRR, we can find the model that can work well with both relation prediction and entity ranking. However, the performance of entity ranking is slightly decreased. To overcome that, the combination between overall MRR and hybrid training can brings huge impact, it's not only improve the relation prediction performance but also overcome the trade-off of overall MRR. 


\section{Future Work}
\label{sec:future}

Our work clearly has some limitations, we haven't consider the HPC with including the best configuration form \cite{chen2021relation}'s HPC space which we may found better models there. 

Secondly, the thesis is only focus on ComplEx model and on only two dataset FB15K-237 and WN18RR, Therefore it would be more interesting if we can 


Furthermore, we observed that reciprocal relation could bring a nice improvement for model's performance on entity ranking. However, when coming to relation ranking with reciprocal relation, an open question would be that $()$ which relation should we consider when evaluation an triple $(i, ?, j)$. One potential solution would be combine reciprocal relation and relation together to have one single relation when performing relation prediction. 


\section{Summary}
In sum, even with the limitations, this study is able to: show importance of relation prediction not only in training objective but also in model selection. 
 
